\documentclass[12pt]{article}

\usepackage[utf8]{inputenc}
\usepackage[french]{babel}
\usepackage{graphicx}
\usepackage{listings}
\usepackage{float}
\usepackage[justification=centering]{caption}
\usepackage[toc,page]{appendix}
\usepackage{url}

\title{Projet PNL}
\date{8 juin 2020}

\begin{document}


\begin{titlepage}
	
\newcommand{\HRule}{\rule{\linewidth}{0.5mm}} % Defines a new command for the horizontal lines, change thickness here

\center % Center everything on the page

%----------------------------------------------------------------------------------------
%	HEADING SECTIONS
%----------------------------------------------------------------------------------------

\textsc{\Large Projet PNL 2019/2020}\\[0.5cm] % Major heading such as course name
\textsc{\large }\\[0.5cm] % Minor heading such as course title

%----------------------------------------------------------------------------------------
%	TITLE SECTION
%----------------------------------------------------------------------------------------

\vspace{0.5cm}
{ \LARGE \bfseries \verb|ouichef_fs| – le système de fichiers le plus classe du monde}\\[0.4cm] % Title of your document
\vspace{1.5cm}

%----------------------------------------------------------------------------------------
%	AUTHOR SECTION
%----------------------------------------------------------------------------------------

\begin{minipage}{0.4\textwidth}
	\begin{flushleft} \large
		\emph{Auteurs:}\\
		Pierre-Loup \textsc{Gosse}\\
		Salim Toufik \textsc{Nedjam}\\
		Youcef \textsc{Mcirdi}
	\end{flushleft}
\end{minipage}
~
\begin{minipage}{0.4\textwidth}
	\begin{flushright} \large
		\emph{Encadrants:} \\
		Redha \textsc{Gouicem}\\
		Rémi \textsc{Oudin} \\
		Julien \textsc{Sopena}
	\end{flushright}
\end{minipage}\\[2cm]


%----------------------------------------------------------------------------------------
%	DATE SECTION
%----------------------------------------------------------------------------------------

{\large 8 Juin 2020}\\[2cm] % Date, change the \today to a set date if you want to be precise

%----------------------------------------------------------------------------------------
%	LOGO SECTION
%----------------------------------------------------------------------------------------

\includegraphics[scale=0.35]{include/logo_sorbonne.png}  % Include a department/university logo - this will require the graphicx package
\hspace{1.5cm}
\includegraphics[scale=0.35]{include/logo_lip6.png}\\ % Include a department/university logo - this will require the graphicx package

\vfill % Fill the rest of the page with whitespace
	
\end{titlepage}

\tableofcontents
\newpage

\section{Implémentation}

Le mécanisme de rotation de fichier s'exécute lorsqu'une des deux conditions est vérifiée.
Notamment lors de la création d'un fichier dans la fonction \verb|ouichefs_create|, mais aussi
lors de l'écriture d'un fichier dans la fonction \verb|ouichefs_write_begin|. 
C'est la fonction \verb|ouichefs_fblocks| sera appelée avec comme paramètre l'inode du dossier
source où il faut libérer des blocs.
\\

Notre conception reste assez simple mais assure un bon fonctionnement.
La fonction \verb|ouichefs_fblocks| à pour rôle de trouver une victime à partir de l'inode
du dossier source et de la supprimer.
Pour ce faire nous avons découpé notre mécanisme en plusieurs parties afin de le rendre plus
souple et générique.

\subsection{Mécanisme de rotation}

\subsubsection{Structure}
La suppression d'une inode victime se fera par l'utilisation de fonction déjà existante,
demandant de connaître l'inode du dossier parent de la victime.
L'ajout d'une structure de parenté a été ajouté:
\begin{lstlisting}[tabsize=4]
struct ouichefs_inode_kinship {
	struct inode *parent;
	struct inode *inode;
};
\end{lstlisting}

\subsubsection{Stratégie}
Pour permettre à notre mécanisme d'être modulaire nous avons implémenté une fonction de
comparaison, dite de stratégie, qui permet de comparer deux inodes en fonction de leurs champs.

\begin{lstlisting}[tabsize=4]
int (*ouichefs_fblocks_strategy) (struct inode *a, 
	struct inode *b) = ouichefs_fblocks_strategy_mtime;
	
EXPORT_SYMBOL_GPL(ouichefs_fblocks_strategy);
\end{lstlisting}

La stratégie appliquée par défaut est \verb|ouichefs_fblocks_strategy_mtime| qui compare
les deux inodes en fonction de leur date de modification.
Si la fonction retourne un résultat supérieur à 0 alors l'inode passée en second paramètre
sortira gagnante de la comparaison, et sera ici considéré comme la victime.

\subsubsection{Action}
Une fonction dite d'action se chargera pour une inode donnée de déterminer si elle est la
nouvelle victime après une comparaison avec la stratégie mis en place.

\begin{lstlisting}[tabsize=4]
void ouichefs_fblocks_action(struct inode *dir,
	struct inode *inode,
	void **data)
{
	struct ouichefs_inode_kinship **victim;
	int ret = 0;
	
	victim = (struct ouichefs_inode_kinship **) data;
	
	if ((*victim)->inode == NULL)
		ret = 1;
	else if (ouichefs_fblocks_strategy != NULL)
		ret = ouichefs_fblocks_strategy((*victim)->inode, inode);
	
	if (ret > 0) {
		(*victim)->parent = dir;
		(*victim)->inode = inode;
	}
}
\end{lstlisting} 

\subsubsection{Itération}
Une fonction générique d'itération permet d'itérer sur tous les inodes d'un inode source.
Cette fonction prend en paramètre la fonction \verb|action| à appliquer à chaque inode représentant
un fichier, ainsi qu'une variable \verb|data| qui sera fournie à la fonction action et qui
représente ici la victime.

\begin{lstlisting}[tabsize=4]
void ouichefs_iterate(struct inode *dir,
		void (*action)(struct inode *dir, 
				struct inode *inode, void **data),
		void **data);
\end{lstlisting}

\subsubsection{Libération de blocs}
C'est la fonction \verb|ouichefs_fblocks| qui se chargera de lancer l'itération sur les
inodes du dossier source. Elle appellera \verb|ouichefs_iterate| avec comme paramètre
la fonction \verb|ouichefs_fblocks_action| et un pointeur vers la structure de la victime.

Une fois l'itération effectuée la victime contiendra l'inode que nous devons supprimer afin de libérer des blocs.
Cependant l'inode peut avoir un \verb|dentry| qui lui est lié:
\begin{lstlisting}[tabsize=4]
dentry = d_find_any_alias(victim->inode);
\end{lstlisting}

Si \verb|dentry| ne vaut pas \verb|NULL| alors sa suppression se fait en utilisant \verb|vfs_unlink|,
la même fonction utilisée lors d'un appel système \verb|unlink|.
Sinon nous supprimons directement l'inode avec \verb|ouichefs_remove|, une version de \verb|ouichefs_unlink|
qui ne prend pas en paramètre de \verb|dentry| pour l'inode.

\subsection{Politique de suppression avec module}

La variable \verb|ouichefs_fblocks_strategy| étant exportée la modification de la politique de suppression
est un jeu d'enfant. L'insertion du module \verb|ouichefs_strategy_changer| permet de modifier
la stratégie avec une comparaison sur le taille des fichiers.

\subsection{Interaction user / fs}

Nous avons décidé d'implémenter l'interaction user / fs avec le système d'ioctl.

\section{Fonctionnalités implémentées}

\begin{itemize}
	\item Déclenchement du mécanisme lors de la création d'un fichier dans un répertoire plein.
	
	\item Déclenchement du mécanisme lors du déplacement d'un fichier dans un répertoire plein.
	
	\item Déclenchement du mécanisme lors de l'écriture d'un fichier dans un répertoire où le stockage
	à atteint sa limite.
	
	\item Suppression du fichier correspondant à la stratégie mis en place.
	
	\item Modification de la stratégie avec insertion de module.
	
	\item Déclenchement manuel du mécanisme de rotation avec ioctl.
\end{itemize}

\section{Fonctionnalités à travailler}

\begin{itemize}
	\item La fonction d'itération sur les inodes.
\end{itemize}

\section{Fonctionnalités non implémentées}

\begin{itemize}
	\item La détection de l'utilisation de l'inode par un autre processus.
\end{itemize}

\end{document}